\chapter{Introducci�n} 
\label{chap:intro}

\vspace{-0.2cm}

\lsection{Motivaci�n del proyecto}
Ejemplo de referencia a la bibliograf�a~\cite{article:Ejemplo}.

Ejemplo de imagen:
\begin{figure}[h]
  \centerline{
    \mbox{\includegraphics[width=3.00in]{images/logo_eps.eps}}
  }
  \caption{Ejemplo pie de figura 1}
  \label{fig:norm_Daugman}
\end{figure}

\section{Objetivos y enfoque}

\section{Metodolog�a y plan de trabajo}
\subsection{Metodolog�a}

\subsection{Plan de Trabajo}
\subsubsection{Sprint 1 (18/9/2018 - 3/10/2018)}
El primer sprint ha estado centrado tanto en definir con m�s exactitud la direcci�n del proyecto como en un primer acercamiento a las principales herramientas con las que va a desarrollarse. 

Tras unos primeros pasos con Galaxy \cite{1} y Docker \cite{2}, se ha tomado como referencia el trabajo Bioinfworkflow de Sergio Chico \cite{3} como base para la imagen Docker del proyecto. Dado que el proyecto de Github daba algunos problemas en la instalaci�n, se ha desarrollado un script propio que produce los mismos resultados.

Una vez se ha tenido disponible la imagen de Docker, el sprint se ha centrado en algunos aspectos importantes para partes futuras del desarrollo. Entre ellos destaca la investigaci�n acerca del formato de los workflows de Galaxy (.ga) ya que en un futuro ser� necesario generar este tipo de ficheros para introducirlos en Galaxy. Tambi�n resulta relevante la investigaci�n acerca de las posibilidades que ofrece la API de Galaxy \cite{3} y su utilidad en Bioblend \cite{4}, que nos facilitan la opci�n de utilizar Galaxy sin necesidad de hacerlo a trav�s de su interfaz.

\subsubsection{Sprint 2 (4/10/2018 - 17/10/2018)}
La primera semana de este sprint ha estado dirigida a lograr una imagen Docker de Galaxy que contenga un set de herramientas b�sicas para formar un primer workflow. Se han valorado varias opciones de instalaci�n en las que se han utilizado tanto la imagen b�sica de Galaxy \cite{6} como la imagen de Bioinfworkflow \cite{3}. Finalmente se ha optado por utilizar Bioinforworkflow ya que parte de las herramientas necesarias ya estaban incluidas. 
Para realizar esta tarea se ha creado un nuevo fichero Dockerfile as� como el listado de herramientas necesarias para su instalaci�n.

Durante la segunda semana se ha realizado una comparativa entre las diferentes librer�as que permiten desarrollar interfaces visuales en Python, priorizando aquellas que permitiesen un soporte multiplataforma. Finalmente se ha optado por utilizar QtPy5 \cite{7} , dada su sencillez y su soporte multiplataforma.

En el tiempo restante de la segunda semana se han realizado diferentes pruebas con las herramientas de galaxy con las que se formar� el pr�ximo workflow.

\newpage \thispagestyle{empty} % P�gina vac�a 