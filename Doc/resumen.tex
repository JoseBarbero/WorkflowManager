\chapter*{Resumen}

\section*{Resumen}
El auge de las nuevas técnicas de secuenciación masiva está generando una necesidad de análisis de gran cantidad de datos que provoca el acercamiento a la bioinformática de nuevos grupos de investigación antes ajenos a ella. Grupos que solían encargar los análisis de sus datos a otros grupos de investigación están comenzando a desarrollar sus propios workflows para tratar los datos personalmente. 

En este contexto, el grupo de investigación Tecnofood de la Universidad de Burgos, con una larga trayectoria en investigación sobre las fuentes de contaminación en la cadena alimentaria, ha decidido desarrollar un workflow destinado al análisis de la bacteria \textit{Campylobacter jejuni}. Este trabajo estudia su proceso de creación, orientado a establecer una primera herramienta accesible al grupo con la que tendrán la posibilidad de realizar análisis, tanto de las cepas de esta bacteria, como de otras similares en un futuro. La interfaz gráfica de la infraestructura utilizada para su soporte, \textit{Galaxy}, hace posible que el workflow pueda sufrir modificaciones de manera sencilla en un futuro. Esto permitirá dar lugar a variaciones del workflow sin necesidad de un nuevo desarrollo poco accesible a usuarios ajenos a la programación.

El sistema se ha desplegado sobre un contenedor \textit{Docker} con la intención de ahorrar trabajo al usuario evitando la instalación y configuración de un entorno apropiado.

El workflow consta de 6 utilidades principales que resuelven los aspectos de filtrado de calidad (\textit{Trimmomatic} y \textit{Prinseq}), ensamblado (\textit{SPAdes}), etiquetado (\textit{Prokka}), análisis de resistencia a antibióticos (\textit{ABRicate}) y análisis pangenómico (\textit{Roary}). 

Finalmente, el funcionamiento de este workflow se ha puesto a prueba utilizando cuarenta y seis cepas de \textit{Campylobacter jejuni}, secuenciadas con un equipo \textit{Illumina}, procedentes de un matadero del norte de España.

\section*{Palabras Clave}
Bioinformática, \textit{Campylobacter}, workflow, ensamblado, etiquetado, análisis pangenómico, \textit{Illumina}

\newpage

%-------------------------------------------------------------------------------------------------------------------------------------
\section*{Abstract}
The rise of NGS techniques is generating a need for large amount of data to be analysed. This is encouraging new research groups to approach to the field of bioinformatics when they were previously unrelated to it. Groups that used to commission the analysis of their data to other research groups are beginning to develop their own workflows in order to personally analyse this data.

In this context, Tecnofood, a University of Burgos research group with a long research history on sources of contamination in the food chain, has decided to develop a workflow to analyse \textit{Campylobacter jejuni} bacterium. This work studies its creation process, oriented to establish a first tool accessible to the group with which they will have the opportunity to make analyses of the strains of this bacterium and of other ones in the future. The interface of the framework used in its deployment, \textit{Galaxy}, makes the workflow able to easily undergo modifications in the future. This will allow non-programmers to make workflow variations without the need for a new whole development.

The system has been deployed on a \textit{Docker} container trying to avoid unnecessary configuration and installation processes of a new environment to the user.

The workflow is made of 6 main tools that solve aspects of quality filtering (Trimmomatic and Prinseq), assembly (SPAdes), labelling (Prokka), analysis of antibacterial resistance (ABRicate) and pangenomic analysis (Roary).

Finally, this workflow has been tested using forty-six strains of \textit{Campylobacter jejuni}, previously sequenced with an \textit{Illumina} HiSeq and collected in a slaughterhouse in northern Spain.

\section*{Keywords}
Bioinformatics, \textit{Campylobacter}, workflow, assembly, annotation, pangenome analysis, \textit{Illumina}
