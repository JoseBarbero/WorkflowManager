%% Los margenes, tipo de hoja y estilo BOOK
\documentclass[a4paper,11pt,twoside,openright,titlepage]{book}
\usepackage[a4paper,left=1in,right=1in,top=0.6in]{geometry}

\usepackage[T1]{fontenc}    %Ineterprete de t�ldes
\usepackage[latin1]{inputenc}
%\usepackage[Latin1]{inputenc}
\usepackage{amsmath,amssymb}    %Paquete de entornos matematicos

\usepackage{natbib}
\usepackage[english,spanish]{babel}
\selectlanguage{spanish} 
\usepackage{graphicx}
\usepackage{psfrag}
\usepackage{quotchap}
\usepackage{epsfig}
\usepackage[all]{xy}

\usepackage{epsfig}
\usepackage{makeidx}
\usepackage{ifthen}
\usepackage{multicolpar}    %Para poner texto en columnas en plan articulo intercalado con texto normal
\usepackage{multicol,multirow}

\usepackage{url}        %Para direcciones web
\usepackage{marvosym}   %Para imprimir el simbolo de \EUR euro
%\usepackage{eurosym}   %Para imprimir el simbolo de \euro euro
\usepackage{fancybox}   %Para tablas con bordes redondeados


%% Modificaci�n de la plantilla para adaptarla a los requisitos de PFC
\usepackage{fancyhdr}
\pagestyle{fancy}
%%% Cabeceras y pies de p�gina
\fancyhead[CE,CO]{\emph{\titulo}}
\fancyhead[LE,LO,RE,RO]{}
\fancyfoot[LE,RO]{\thepage}
\fancyfoot[CE,CO]{\leftmark}

\renewcommand{\footrulewidth}{.6pt}


%Definiciones de funciones para los titulos
\newlength\salto
\setlength{\salto}{3.5ex plus 1ex minus .2ex}

\newlength\resalto
\setlength{\resalto}{2.3ex plus.2ex}

\newcommand{\lsection}[1]
                {\section{#1}
                \vskip-.9\resalto   %%%% Aqu� reculo el posible salto por defecto de \section
                \hrule
                \vskip+.9\salto}  %%%% vuelvo ha realizar el salto (puedes poner otra vez el 90%)


%Para im�genes de entornos est�ticos \captionFigure{Texto Caption}{Texto Label}
\newcommand{\captionFigure}[2]{
    \refstepcounter{figure}
    \centerline{Figura \thefigure: #1 \label{#2}}
    \addcontentsline{lof}{section}{\thefigure.\ #1\label{#2}}
}

%Para im�genes de entornos est�ticos \NOcaptionFigure{Texto Caption}{Texto Label} "No escribe el caption"
\newcommand{\NOcaptionFigure}[2]{
    \refstepcounter{figure}
    \addcontentsline{lof}{figure}{\thefigure.\ #1\label{#2}}
}


%% Datos del PFC
\newcommand{\titulo}{T\'itulo del TFM}
\newcommand{\autor}{Autor: Nombre Apellido1 Apellido2}
\newcommand{\director}{Nombre Apellido1 Apellido2}
\newcommand{\tutor}{Tutor: Nombre Apellido1 Apellido2}
\newcommand{\ponente}{Ponente: Nombre Apellido1 Apellido2}
\newcommand{\vocal}{Nombre Apellido1 Apellido2}
\newcommand{\vocalsup}{Nombre Apellido1 Apellido2}
\newcommand{\presidente}{Nombre Apellido1 Apellido2}
\newcommand{\presidentesup}{Nombre Apellido1 Apellido2}
\newcommand{\fecha}{MES 20xx}
\newcommand{\carrera}{M\'aster en ...}

\begin{document}
\setlength{\baselineskip}{18pt}  %% Espacio interlinea
\setlength{\parskip}{6pt plus 1pt minus 1pt} %% Espacio interp�rrafo

\input{portada}

\frontmatter %Define el cuerpo inicial del libro en numeraci�n con letras romanas

\input{primera_pag}

\chapter*{Resumen}

\section*{Resumen}


\section*{Palabras Clave}

\newpage

%-------------------------------------------------------------------------------------------------------------------------------------
\section*{Abstract}


\section*{Keywords}


\input{agradecimientos}

\tableofcontents

\newpage \thispagestyle{empty} % P�gina vac�a

\addcontentsline{toc}{chapter}{�ndice de Figuras}    %Para que aparezca en el �ndice
\renewcommand{\listfigurename}{�ndice de Figuras} 
\listoffigures

\newpage \thispagestyle{empty} % P�gina vac�a

\addcontentsline{toc}{chapter}{�ndice de Tablas}    %Para que aparezca en el �ndice
\renewcommand{\listtablename}{�ndice de Tablas} 
\listoftables

\newpage \thispagestyle{empty} % P�gina vac�a

\mainmatter %Define el cuerpo principal del libro numeraci�n normal.

% \input{preambulo}

\chapter{Introducci�n} 
\label{chap:intro}

\vspace{-0.2cm}

\lsection{Motivaci�n del proyecto}
Ejemplo de referencia a la bibliograf�a~\cite{article:Ejemplo}.

Ejemplo de imagen:
\begin{figure}[h]
  \centerline{
    \mbox{\includegraphics[width=3.00in]{images/logo_eps.eps}}
  }
  \caption{Ejemplo pie de figura 1}
  \label{fig:norm_Daugman}
\end{figure}

\section{Objetivos y enfoque}

\section{Metodolog�a y plan de trabajo}
\subsection{Metodolog�a}

\subsection{Plan de Trabajo}
\subsubsection{Sprint 1 (18/9/2018 - 3/10/2018)}
El primer sprint ha estado centrado tanto en definir con m�s exactitud la direcci�n del proyecto como en un primer acercamiento a las principales herramientas con las que va a desarrollarse. 

Tras unos primeros pasos con Galaxy \cite{1} y Docker \cite{2}, se ha tomado como referencia el trabajo Bioinfworkflow de Sergio Chico \cite{3} como base para la imagen Docker del proyecto. Dado que el proyecto de Github daba algunos problemas en la instalaci�n, se ha desarrollado un script propio que produce los mismos resultados.

Una vez se ha tenido disponible la imagen de Docker, el sprint se ha centrado en algunos aspectos importantes para partes futuras del desarrollo. Entre ellos destaca la investigaci�n acerca del formato de los workflows de Galaxy (.ga) ya que en un futuro ser� necesario generar este tipo de ficheros para introducirlos en Galaxy. Tambi�n resulta relevante la investigaci�n acerca de las posibilidades que ofrece la API de Galaxy \cite{3} y su utilidad en Bioblend \cite{4}, que nos facilitan la opci�n de utilizar Galaxy sin necesidad de hacerlo a trav�s de su interfaz.

\subsubsection{Sprint 2 (4/10/2018 - 17/10/2018)}
La primera semana de este sprint ha estado dirigida a lograr una imagen Docker de Galaxy que contenga un set de herramientas b�sicas para formar un primer workflow. Se han valorado varias opciones de instalaci�n en las que se han utilizado tanto la imagen b�sica de Galaxy \cite{6} como la imagen de Bioinfworkflow \cite{3}. Finalmente se ha optado por utilizar Bioinforworkflow ya que parte de las herramientas necesarias ya estaban incluidas. 
Para realizar esta tarea se ha creado un nuevo fichero Dockerfile as� como el listado de herramientas necesarias para su instalaci�n.

\subsubsection{Sprint 3 (18/10/2018 - 31/10/2018)}
El sprint ha estado centrado en la correcta ejecuci�n del workflow con las herramientas iniciales desde Galaxy. Durante el proceso de configuraci�n han surgido varias complicaciones que han impedido terminar el workflow completo en este sprint. 
En un principio han surgido problemas con el filtrado de calidad utilizando Prinseq. Este problema no ha llegado a ser resuelto en este sprint a falta de tratar el tema con el grupo de Tecnofood.
A continuaci�n se encontraron ciertos problemas con el formato de salida de la herramienta Prokka. A pesar de que la salida est� marcada como formato gff3, un par�metro interno lo etiquetaba como gff. Esto imped�a que la salida de Prokka fuese introducida como input en las herramientas siguientes.

Dados estos errores, se decidi� trabajar en paralelo con la API de Galaxy desde Python para intentar ejecutar tanto las herramientas como el workflow de una manera menos restringida. Finalmente se ha llevado a cabo el desarrollo necesario para subir los ficheros a un historial y ejecutar cada una de las herramientas del workflow desde Python.

\newpage \thispagestyle{empty} % P�gina vac�a 

\chapter{Estado del arte}
\label{chap:estadodelarte}

\section{Técnicas en bioinformática}

\subsection{Control de calidad}
Las nuevas técnicas de secuenciación masiva proporcionan una gran ventaja en cuanto a la velocidad de lectura de las secuencias~\cite{ngs}. Sin embargo, hay que tener especial precaución en el análisis de la calidad de estas, ya que una mayor velocidad puede llegar a implicar errores en las lecturas en algunos casos. Especialmente, en aquellos en los que se va a realizar un análisis en el que se pueden transmitir estos errores a través de cada una de las herramientas que componen el proceso, pudiendo llegar a provocar conclusiones erróneas al final del análisis.

Uno de los principales factores a tener en cuenta es el método de secuenciación que se haya seguido para obtener las lecturas. Por ejemplo, la tecnología de \textit{Illumina} presenta ciertas complicaciones a raíz de la utilización de adaptadores y primers durante la secuenciación. Al comienzo de ese proceso, durante la preparación de las secuencias, se requiere que se unan a la placa se secuenciación. Para ello, se utilizan adaptadores: cadenas de nucleótidos que se unen a los dos extremos de la secuencia, permitiendo el enlace con los adaptadores complementarios existentes en la placa. Además, para multiplicar el número de secuencias disponibles, se realiza un proceso de PCR, que consiste en replicar las secuencias utilizando polimerasa (la enzima encargada de la replicación y transcripción de ADN). Para que la polimerasa comience este proceso, necesitará reconocer unos patrones concretos en la secuencia de ADN, denominados primers, que deberán haber sido añadidos previamente.

La contaminación con adaptadores y primers utilizados en el proceso de secuenciación, los artefactos (uniones inespecíficas), las lecturas de baja calidad, las inserciones y deleciones son varias de las causas que pueden llevar a análisis problemáticos~\cite{plosone}. No obstante, en la actualidad, existen numerosos métodos enfocados al tratamiento de estos problemas, normalmente agrupados en herramientas entre las que destacan algunas como \textit{FastQC}~\cite{fastqc} o \textit{Prinseq}~\cite{schmieder_prinseq}.

\subsection{Ensamblado}
Los sistemas de secuenciación actuales pierden calidad cuando aumenta el tamaño de las lecturas, por lo tanto, se generan gran cantidad de pequeñas lecturas que deberemos unir. El objetivo del proceso de ensamblado es recoger todos estos fragmentos menores que conformarían una secuencia mayor para alinearlos y empalmarlos con el objetivo de reconstruirla~\cite{assembly_wiki}. Dada la tendencia al aumento de la cantidad de datos a procesar por los ensambladores a raíz de las nuevas técnicas de secuenciación, estos se han visto empujados a emplear técnicas cada vez más eficientes para evitar crear un cuello de botella en esta fase del análisis. Los diferentes aspectos de cada secuenciación como el tamaño, si se trata de un organismo procariota o eucariota, de un transcriptoma, metagenoma, etc, influyen en la eficacia de unos u otros algoritmos. En nuestro caso, al tratarse de bacterias, cuyo genoma es pequeño, se ha decidido utilizar una herramienta destinada a esta característica, como es \textit{SPAdes}~\cite{Nurk2013}. 

Hay que destacar que el problema del ensamblado es, computacionalmente, extremadamente costoso, por lo que la importancia de encontrar un algoritmo que resuelva el problema en un tiempo aceptable es enorme. En este caso, la tecnología en la que se basa tanto esta herramienta como otras muy potentes, como \textit{Velvet}~\cite{Zerbino2008} son los grafos de De Bruijn~\cite{Compeau2011}. Se trata de un tipo de grafos utilizados para representar solapamientos entre secuencias. Es importante tener en cuenta que, en esta representación, los enlaces del grafo estarán formados por las subcadenas de tamaño \textit{k}, denominadas \textit{k-mers} de cada fragmento, mientras que los nodos serán las uniones que utilicen las dos subcadenas para lograr empalmar el final de una con el inicio de la otra, como se aprecia en la figura~\ref{fig:DeBruijn}. La base del algoritmo centrado en este tipo de grafos es encontrar su camino hamiltoniano (que recorra todos los nodos de un grafo una sola vez), siendo este camino la secuencia ensamblada completa.

\begin{figure}
    \begin{center}
      \includegraphics[scale=0.5]{images/DeBruijnGraph.png}
      \caption{Ejemplo de grafo De Bruijn de tamaño de  k-mer 3}
      \label{fig:DeBruijn}
    \end{center}
\end{figure}

\subsection{Anotación}
El proceso de anotación del genoma se basa en identificar y etiquetar todos los aspectos relevantes encontrados en una secuencia, incluyendo, principalmente, regiones codificantes; aunque también son interesantes otros elementos como ARN no codificante, péptidos señal, etc~\cite{RichardsonWatson2012}. 

En algunos casos el proceso se basa en la ejecución de un pipeline automático de anotación seguido de un proceso de filtrado manual~\cite{Stothard2006}. El objetivo en este proyecto es la automatización del proceso en un tiempo de ejecución asequible sin dejar de lado la precisión. En experimentos realizados, \textit{Prokka} ha dado mejores resultados que alternativas como \textit{RAST} o \textit{xBase2} y ha demostrado ser eficiente incluso en ordenadores de usuario con la potencia típica de un PC de sobremesa~\cite{Seemann2014}.

\textit{Prokka}~\cite{Seemann2014} basa su anotación en una serie de herramientas  externas, cada una de ellas especializada en la identificación de una característica de la secuencia concreta, como se explica en la tabla~\ref{table:ProkkaTools}.

\begin{table}[!htb]

\begin{center}
\begin{tabularx}{\textwidth}{bb}
\arrayrulecolor{NavyBlue}\hline
\textbf{\textcolor{NavyBlue}{Herramientas}} &
\textbf{\textcolor{NavyBlue}{Características objetivo}}\\
\quad Prodigal~\cite{Hyatt2010} &
\begin{minipage}[t]{\linewidth}
\quad Secuencias codificantes (CDS)
\end{minipage}\\

\quad RNAmmer~\cite{Lagesen2007} &
\begin{minipage}[t]{\linewidth}
\quad ARN Ribosómico (rRNA)
\end{minipage}\\

\quad Aragorn~\cite{Laslett2004} &
\begin{minipage}[t]{\linewidth}
\quad ARN de transferencia
\end{minipage}\\

\quad SignalP~\cite{Petersen2011} &
\begin{minipage}[t]{\linewidth}
\quad Péptidos señal
\end{minipage}\\

\quad Infernal~\cite{Kolbe2011} &
\begin{minipage}[t]{\linewidth}
\quad ARN no codificante
\end{minipage}\\
\hline
\end{tabularx}
\end{center}
\caption{Herramientas externas de \textit{Prokka}}
\label{table:ProkkaTools}
\end{table}

\subsection{Análisis de resistencia a antibióticos}
Desde el comienzo de la utilización de antibióticos por parte del ser humano, se ha sometido a las bacterias a un proceso de selección natural por el cual aquellas con la capacidad de sobrevivir a cierta concentración de antibiótico, proliferarán. La resistencia a antibióticos se está convirtiendo en uno de los problemas más amenazantes para el ser humano~\cite{Neu1064}. \textit{Campylobacter jejuni}, la bacteria que nos ocupa en este caso, no es una excepción y ya se han detectado casos de resistencia en ella~\cite{Smith1999}. 

Comienzan a identificarse algunos de los genes responsables de esta resistencia y se están creando bases de datos con las que registrarlos~\cite{ARDB}. Consecuentemente, han ido surgiendo herramientas basadas en estas bases de datos con las que contrastar si los fragmentos de cierta secuencia están incluidos en una de ellas. \textit{ABRicate}~\cite{seemann_:mag_right:_2019} es una de las muchas que existen e incluye, tanto bases de datos dedicadas a genes de resistencia a antibióticos como bases de datos de genes de virus.

\subsection{Análisis pangenómico}
El análisis pangenómico en bacterias es un proceso complejo, condicionado siempre a la aparición de nuevos genes. No obstante, su resolución puede suponer ventajas muy significativas~\cite{MEDINI2005589}. Modelos matemáticos aplicados a este problema concluyen que genes únicos continuarán apareciendo independientemente de la cantidad de genomas que se secuencien~\cite{Tettelin13950}. Además, la construcción de este genoma supone un coste computacional muy relevante debido a que se trata de un problema NP-complejo~\cite{Nguyen2015}.

Llegados a este punto, el objetivo en este caso es la construcción de un pangenoma a partir de todas las cepas de las que disponemos y realizar un análisis en el que se indique qué cepas contienen cada gen. En la actualidad se están desarrollando herramientas dedicadas a esta tarea. \textit{Roary}~\cite{Page2015} es una de ellas, basada en la utilización de los fragmentos de secuencias codificantes etiquetados por otras herramientas como \textit{Prokka} para transformarlos en secuencias de proteínas y así simplificar el conjunto de datos. De esta manera, la comparación <<todos contra todos>> de \textit{BLASTP}~\cite{Madden}, se ve muy optimizada.

\section{Herramientas en informática}
Además de las herramientas puramente enfocadas al análisis genético, son necesarias otras técnicas más centradas en la informática para facilitar la utilización del workflow. En este caso, una plataforma con la que poder interactuar con él y un sistema en el que sostener toda su infraestructura.
\subsection{Plataforma de soporte del workflow}
Existe una gran variedad de plataformas sobre las que construir y ejecutar un workflow en la actualidad. Uno de los primeros en aparecer fue \textit{Discovery Net}~\cite{Curcin:2002:DNT:775047.775145}, una utilidad enfocada a coordinar trabajos de servicios ejecutados de manera remota. Sin embargo, es un sistema basado en código y no posee su propia interfaz gráfica, lo que dificultaría su utilización por parte de usuarios ajenos a la programación. Lo mismo ocurre con muchos de los sistemas más utilizados, como \textit{Anduril} \cite{Ovaska2010}, \textit{BioQueue}~ \cite{10.1093/bioinformatics/btx403} o \textit{Cuneiform}~\cite{brandt_reisig_leser_2017}. Sin embargo, hay varios frameworks que permiten un control gráfico del workflow, como \textit{Apache Taverna}~\cite{10.1093/nar/gkt328}, \textit{Triana}~\cite{Taylor2007}, \textit{Kepler}\cite{Ludascher:2006:SWM:1148437.1148454}, \textit{Galaxy}~\cite{Galaxy} o \textit{Yawl}~\cite{yawl2004}. Existen varias publicaciones que analizan las ventajas e inconvenientes de cada uno de los sistemas \cite{4786077, Abouelhoda:2010:MPI:1833398.1833400, Nyronen:2012:DII:2361999.2362006, 10.1093/bib/bbw020}. Entre ellos, por su desarrollo más avanzado y su capacidad para desplegar un sistema completo en un navegador, además de una interfaz fácil de utilizar y una API \textit{Python}~\cite{GalaxyAPI} disponible, destaca \textit{Galaxy}.

\subsection{Virtualización}
Llamamos virtualización a un conjunto de tecnologías destinadas a simular componentes hardware en forma de software. Eso permite la creación de entornos en los que podemos disponer de software y hardware dentro de nuestro propio sistema. Por ejemplo, existe la posibilidad de establecer un sistema \textit{Ubuntu} virtual dentro de un sistema Windows antifrión. En el caso de este ejemplo, estaríamos hablando de una virtualización total, en la que se simula el sistema operativo completo, lo que supone una carga muy pesada debido a que existirá una gran cantidad de software que no va a ser utilizado. En los últimos años se han ido mejorando las técnicas de virtualización parcial, en las que es posible contener solamente la parte mínima de sistema operativo y de software necesarios para la herramienta que queramos simular. Estas técnicas se basan en contenedores, en los que se encapsulan las imágenes con las dependencias necesarias para su funcionamiento. Entre las herramientas más utilizadas, destaca \textit{Docker}~\cite{Docker}, siendo la más reconocida por su popularidad entre los usuarios.

Dado que existe una imagen \textit{Galaxy} estable para \textit{Docker} a partir de la cual poder trabajar, así como una capa adicional desarrollada anteriormente para el mismo grupo de investigación, podemos concluir que \textit{Docker} es el sistema que más se adecúa a las necesidades del proyecto.


\newpage \thispagestyle{empty} % Página vacía 

\chapter{Sistema, diseño y desarrollo}
\label{chap:sistemadesarrollado}

\section{Estructura general del sistema}

El sistema desarrollado\footnote{Disponible en \url{https://github.com/JoseBarbero/WorkflowManager}} se ha basado en \textit{Docker} para evitar problemas de configuración y portabilidad, es decir, podrá ser desplegado en cualquier sistema que soporte esta tecnología. La estructura está basada en varias capas, todas ellas sobre el sistema anfitrión que elijamos, pudiendo adaptarse incluso a un servidor. En la figura   \ref{fig:DiagramaDelSistema} se puede apreciar la distribución de las capas mencionadas. 

\begin{itemize}
\item La primera capa de esta estructura es el propio sistema anfitrión que, normalmente, será el equipo de utilice el usuario regularmente. Su única función es la de servir de base para la ejecución de la imagen {\itshape{Docker}}.
\item Inmediatamente por encima del sistema anfitrión se encuentra la imagen {\itshape{Docker}}, encargada de facilitar la configuración y las dependencias necesarias para la ejecución de {\itshape{Galaxy}}.
\item La imagen {\itshape{Docker}} incluye el despliegue de {\itshape{Galaxy}} sobre el puerto 8080. Por lo que podemos considerar como otra capa del sistema a la propia ejecución de {\itshape{Galaxy}}.
\item El punto principal del trabajo se encuentra en el apartado de workflows dentro de {\itshape{Galaxy}}. En él se localiza la secuencia de herramientas que conforma el núcleo del proyecto.
\item Exteriormente a {\itshape{Galaxy}} y {\itshape{Docker}}, se han desarrollado dos herramientas que interactúan con el workflow. La primera es un script {\itshape{Python}} que se encarga de toda la configuración y ejecución del workflow, para facilitar su utilización a usuarios no experimentados en informática. La segunda herramienta se encarga de tratar los datos obtenidos de la ejecución para generar un formato más fácil de comprender y manipular por el usuario.
\end{itemize}

\begin{figure}
    \begin{center}
      \includegraphics[scale=0.45]{images/DiagramaDelSistema.png}
      \caption{Estructura del sistema}
      \label{fig:DiagramaDelSistema}
    \end{center}
\end{figure}
\section{Desarrollo y configuración de \itshape{Docker}}
La imagen \textit{Docker} utilizada está basada, con algunas modificaciones, en la desarrollada por Sergio Chico en su Trabajo de Fin de Máster \cite{Chico2018}. La cual, a su vez, hereda de una imagen \textit{Docker Galaxy} estable desarrollada por Björn A. Grüning \cite{GalaxyDocker}.

Utilizando como punto de partida la imagen de Sergio Chico, se han realizado una serie de adaptaciones para adecuarla al uso de nuestro workflow. La primera modificación realizada fue la instalación de las nuevas herramientas necesarias que no estaban disponibles en ese momento. Al conjunto de herramientas iniciales se añadieron \textit{Prinseq}, \textit{Spades}, \textit{Roary}, \textit{ABRicate} y \textit{Prokka}. Además, en las opciones de instalación de herramientas de \textit{Galaxy} solo estaba disponible para la versión previa de \textit{Roary} que producía, por algún tipo de bug, ficheros de salida vacíos. Por lo tanto, se creó una nueva herramienta a partir de la original con la nueva versión disponible para escritorio de \textit{Roary}.

Una vez realizadas estas adaptaciones a la imagen, se ha dejado disponible en \textit{Dockerhub} bajo el nombre \textit{jbarberoaparicio/workflowmanager}\footnote{https://hub.docker.com/r/jbarberoaparicio/workflowmanager}. 

Además de esto, se han desarrollado varios scripts simples para facilitar el uso de \textit{Docker}. El primero de ellos se encarga del montaje completo de la imagen a partir del \textit{Dockerfile}, en caso de que quisieran realizarse modificaciones en un futuro. El segundo se encarga del arranque de la imagen desplegándola a través de \textit{Docker}. El tercero se desarrolló para resolver un bug en \textit{OSX} por el cual los ficheros eliminados dentro de la imagen no se eliminaban correctamente. Esto provocaba que un fichero tipo \textit{log} ocupase todo el espacio disponible en el disco. El script se encarga de eliminar esta información sin borrar las imágenes existentes.


\section{Desarrollo y configuración del workflow en \itshape{Galaxy}}
A lo largo del workflow, se realizarán varios análisis diferenciados de las secuencias. En algunos casos estos análisis servirán  únicamente de entrada para el siguiente paso y en otros aportarán por sí mismos parte de la información deseada. 
Cabe destacar que se ha definido la entrada del workflow como un formato de dos colecciones, una para cada dirección de las lecturas de las secuencias. 

\subsection{\itshape{Trimmomatic}}
Como se ha explicado en el capítulo \ref{chap:estadodelarte}, la secuenciación con la tecnología \textit{Illumina} implica la utilización de primers y adaptadores que deben ser eliminados para un correcto procesamiento posterior. Para ello, se ha decidido utilizar la herramienta \textit{Trimmomatic}~\cite{Bolger2014}, que incluye diferentes protocolos orientados a diferentes tecnologías de \textit{Illumina} como \textit{MiSeq} o \textit{HiSeq}.

\textit{Trimmomatic} se encargará de recibir los dos ficheros \textit{fastq} iniciales emparejados de cada cepa y les aplicará el protocolo de filtrado adecuado. En este caso se ha utilizado \textit{Nextera}, dado que la secuenciación se llevó a cabo con un equipo \textit{HiSeq}. Una vez eliminadas las secuencias de primers y adaptadores, los nuevos ficheros \textit{fastq} se enviarán a \textit{Prinseq} para realizar a las secuencias un control de calidad.

\subsection{\itshape{Prinseq}}
Con el objetivo de realizar un filtrado de calidad de las secuencias, se ha utilizado \textit{Prinseq}~\cite{schmieder_prinseq}. Esta herramienta, desarrollada en \textit{Perl}, ofrece la posibilidad de generar un informe en el cual se indican diferentes estadísticas sobre la calidad de cierta secuencia. Además, utiliza esa información para realizar el filtrado que le indiquemos, pudiendo ser en función de la longitud de las lecturas o de su calidad.

Como formato de entrada, introduciremos en \textit{Prinseq} un fichero \textit{fastq}. En esta situación, proveniente de los resultados de \textit{Trimmomatic}. Como salida, obtendremos otro fichero \textit{fastq} en el que se habrán eliminado todos los fragmentos que no hayan cumplido las condiciones que se hayan fijado en los parámetros de ejecución. Para nuestro caso, se han eliminado las secuencias de longitud menor que 50 y de índice de calidad menor de 15. El resto de parámetros se han mantenido por defecto a excepción de la indicación de que se trata de una secuencia <<paired-end>>, necesaria para la ejecución de la manera en la que deseamos llevarla a cabo.

\textit{Prinseq} ha sido una de las herramientas que menos complicaciones ha presentado a lo largo del proyecto, devolviendo errores comprensibles cuando las ejecuciones no eran correctas y ofreciendo toda la información necesaria para resolverlo. A pesar de ello, la versión de \textit{Prinseq} disponible en \textit{Galaxy} tiene la desventaja de no generar el informe con todas las estadísticas resultantes que sí genera su versión web.

\subsection{\itshape{SPAdes}}
\textit{SPAdes}~\cite{Nurk2013} ha sido la utilidad seleccionada para la fase de ensamblado. Se trata de un conjunto de herramientas que facilitan un modo de recoger todas las lecturas validadas en el proceso de filtrado y encadenarlas de la manera en la que corresponden en la secuencia. Fue desarrollado enfocándose a genomas pequeños, como bacterias y hongos. Por lo tanto, se adapta bien al caso de uso que estamos tratando.

Para su ejecución, se debe indicar a \textit{SPAdes} la estructura que estamos siguiendo en el workflow, introduciendo ficheros separados en los dos sentidos de lectura de las secuencias en forma de colecciones. Estos ficheros provendrán de la salida de la ejecución de \textit{Prinseq}, tras realizar el filtrado de calidad. El resto de parámetros de la ejecución, a excepción de la longitud de los k-mer, establecida a 77, se han mantenido por defecto. También se le ha indicado que las bibliotecas a utilizar sean <<paired-end>>.

Dado el gran tamaño de las lecturas de los casos en los que el workflow ha sido probado, \textit{SPAdes} ha resultado problemático por tener que almacenar en RAM todas las lecturas al mismo tiempo. Esto ha implicado que, para las muestras de prueba, se haya requerido de 16 GB de memoria RAM para poder ejecutarlo. Además, los mensajes de error de la versión \textit{Galaxy} de \textit{SPAdes} no dan demasiada información en muchos casos. En algunos otros, relacionados con la capacidad de memoria, no se producía ningún mensaje de error, sino que se daba por correcta la ejecución a pesar de que los ficheros retornados estaban vacíos. A pesar de estos problemas de desarrollo, \textit{SPAdes} ha terminado cumpliendo su función correctamente tras identificar los errores del proceso. 

\subsection{\itshape{ABRicate}}
Uno de los principales intereses al comenzar el proyecto era el análisis de resistencia a antibióticos de las cepas. \textit{ABRicate}~\cite{seemann_:mag_right:_2019} es una herramienta desarrollada en \textit{Perl} que permite la localización de genes vinculados a resistencia a antibióticos o relacionados con virus realizando un cribado masivo de los contigs introducidos. Para ello utiliza las bases de datos de \textit{Resfinder}~\cite{Resfinder}, \textit{CARD}~\cite{CARD}, \textit{ARG-ANNOT}~\cite{Gupta212}, \textit{NCBI} \textit{BARRGD}, \textit{EcOH}, \textit{PlasmidFinder}~\cite{Carattoli3895}, \textit{Ecoli\_VF} y \textit{VFDB}~\cite{VFDB}. En nuestro caso, ese input serán los contigs obtenidos por \textit{SPAdes} para cada una de las secuencias. Esta herramienta forma uno de los finales del workflow, es decir, su salida no sirve como entrada a ninguna otra herramienta sino que los datos que se obtienen de ella tienen su utilidad propia.

La configuración de \textit{ABRicate} es muy sencilla y ofrece pocas posibilidades de variación, por lo que no ha sido necesario modificar ningún parámetro de la herramienta. La ejecución por defecto no ha presentado problemas.

Una de las limitaciones de \textit{ABRicate} es que no admite lecturas tipo \textit{fastq}, solamente contigs. En nuestro caso los contigs provienen de la salida de \textit{SPAdes}, por lo que es una herramienta adecuada.

\subsection{\itshape{Prokka}}
Para la fase de anotación se ha decidido utilizar \textit{Prokka}~\cite{Seemann2014}, una herramienta desarrollada en \textit{Perl} destinada específicamente a genomas procariotas y que destaca por su rapidez. El proceso de anotación de \textit{Prokka} se divide en dos fases. En la primera, se utiliza \textit{Prodigal} para la identificación de regiones codificadoras de proteínas. En la segunda, se compara con varias bases de datos para reconocer, por similitud, cuál es la proteína de la que se trata.

La configuración de \textit{Prokka} en \textit{Galaxy} no ha resultado complicada a pesar de que los parámetros son numerosos. El único problema encontrado en la configuración ha sido a raíz del formato \textit{gff3}. A pesar de que se indique que los ficheros de salida siguen esa estructura, realmente no es así, sino que internamente son formato \textit{gff}. Para solucionar este inconveniente se ha añadido una fase nueva en el workflow, en la que una herramienta simple de concatenación une la salida \textit{gff} con la propia secuencia del genoma.

\subsection{\itshape{Roary}}
Uno de los objetivos principales del proyecto era la realización de un análisis pangenómico. Utilizando como entrada las anotaciones provenientes de \textit{Prokka} de todas las muestras, \textit{Roary}~\cite{Page2015} se encarga de calcular el pangenoma. Ha resultado ser una herramienta realmente rápida, con la que en unos minutos se obtienen los resultados.

\textit{Roary} está desarrollado para recibir ficheros \textit{gff3} desde \textit{Prokka}, por lo que no debería haber problemas de compatibilidad (una vez solucionado el error de formato comentado en el punto anterior). El problema con la configuración de \textit{Roary} dentro del workflow es que, si se instala desde el repositorio oficial en el \textit{Tool Shed} de \textit{Galaxy}, no se obtiene ningún tipo de contenido en los ficheros de salida. Estudiando este problema, se llegó a la conclusión de que se podría deber a que la versión disponible en este repositorio no es la versión final publicada para escritorio. Por lo tanto, se creó una herramienta \textit{Galaxy} nueva utilizando la versión standalone de \textit{Roary}. Una vez disponible esta herramienta en \textit{Galaxy} y manteniendo la configuración establecida para la versión anterior, los ficheros fueron obtenidos correctamente.

\textit{Roary} recibe todos los ficheros tratados por el workflow (a excepción de la salida independiente de \textit{ABRicate}) y sirve como herramienta final. La clasificación de cada cepa según su mejor coincidencia para cada gen con el pangenoma creado por \textit{Roary} no son enviados a ninguna otra herramienta dentro de \textit{Galaxy} y finalizan el workflow.


\section{Desarrollo de la capa externa con \itshape{Python}}
A pesar de que el workflow completo ha sido desarrollado dentro de \textit{Galaxy}, se han añadido algunas funcionalidades extra utilizando \textit{Python}.

\subsection{Capa de uso simplificado}
Dado que el equipo al que está destinada la herramienta no pertenece al campo de la informática, se solicitó una herramienta que fuese sencilla de utilizar. Por ello, se ha creado un script que realiza todos los pasos necesarios para ejecutar el workflow simplemente con ejecutarlo, sin ser necesario ningún tipo de configuración. En caso de que el usuario tuviese conocimientos técnicos, seguirá teniendo la posibilidad de utilizar la herramienta sin este script.

Para ello, se ha utilizado la API de \textit{Galaxy}, parte de la librería \textit{BioBlend}\footnote{\url{https://bioblend.readthedocs.io/en/latest/}}. Esta API nos permite controlar todos los aspectos de una instancia \textit{Galaxy} desde código \textit{Python}.

Desde el script se crea una sesión en la instancia de \textit{Galaxy} a partir de la cual se realizan todas las tareas. Una vez conectado, el script carga el archivo que almacena el workflow que va a ejecutar. Después, para mantener los datos aislados e identificados, crea historiales nuevos para los ficheros de entrada y los de salida con un nombre único a partir de la fecha y hora de creación. A continuación carga los datos brutos de inicio, lo que, dependiendo de su tamaño, puede llevar unos minutos. Para que el workflow funcione correctamente, necesita recibir los datos en forma de colección, por lo que el script se encarga de crear una para cada conjunto de datos, es decir, lecturas en un sentido y en el inverso. El paso siguiente, la ejecución del workflow, es el fundamental y el más costoso en tiempo. Para controlar el estado de la tarea en cada momento, se va realizando un control cada pocos segundos que comprueba si las tareas siguen en proceso. Una vez finalizado, los datos de salida se mantienen en un historial propio, pero para poder utilizarlos fuera de \textit{Galaxy}, el script se encarga de almacenarlos en el disco local del equipo en una carpeta con el nombre del historial.

\subsection{Agrupación de resultados}
Para facilitar el acceso a los resultados de todas las herramientas, se ha planteado la creación de un solo fichero que pueda abrirse con \textit{Excel}, herramienta que resulta familiar al grupo que utilizará el workflow. Para ello, desde un script de \textit{Python} se genera un solo fichero en el que cada hoja contiene los resultados de uno de los pasos del workflow.

\chapter{Experimentos Realizados y Resultados}
\label{chap:experimentos}

\lsection{Escenario de pruebas}

\lsection{Caso real: análisis de \textit{Campylobacter jejuni}}




\chapter{Conclusiones y trabajo futuro}
\label{chap:conclusiones}
Tomando como referencia los objetivos planteados en el capítulo \ref{chap:intro}, podemos concluir que las metas establecidas en este proyecto se han cumplido. El resultado del desarrollo ha sido una aplicación funcional que ha podido ser puesta a prueba con un caso real.

El primero de los objetivos se centraba en permitir una forma de instalación y configuración limpia de cara al usuario. La adaptación de una la imagen \textit{Docker} de \textit{Galaxy} ya disponible ha presentado varios problemas. Uno de ellos provocaba que la ejecución superase las capacidades del procesador y este fallase. La resolución de este tipo de inconvenientes así como la introducción de algunas nuevas funcionalidades y configuraciones han terminado proporcionando una imagen estable con la que el usuario pueda trabajar sin inconvenientes.

Uno de los objetivos más relevantes era la creación del propio workflow. La configuración de las herramientas no ha presentado problemas graves más allá de los problemas de ejecución a raíz de la configuración de \textit{Docker}. Tras numerosas pruebas de funcionamiento, el workflow fue mejorado incluyendo las utilidades planteadas en los objetivos. La creación de scripts encapsulando órdenes que los usuarios repetirán en numerosas ocasiones facilitará el manejo del workflow y acercará su uso a quien esté menos familiarizado con la programación de estas órdenes.

El desarrollo de una herramienta desde cero implica que, una vez completada y funcional, se abren varias lineas de trabajo futuras a través de las cuales ampliar o mejorar el desarrollo. Una de las posibles líneas podría centrarse en pulir algunos aspectos del comportamiento interno del workflow en \textit{Galaxy}. Por ejemplo, varios de los nombres de los datos generados podrían ser más descriptivos utilizando el nombre de la cepa inicial. Respecto a las funcionalidades, también podría abrirse una línea en la que añadir nuevas utilidades relacionadas con los ficheros obtenidos. Se trata de una opción totalmente abierta con un gran número de posibilidades. También, debido a la facilidad que presenta \textit{Galaxy} para añadir nuevas herramientas, se podría ampliar la funcionalidad de esta manera sin provocar demasiadas consecuencias en el resto del funcionamiento.

Por último, dado que el proyecto se enfoca a usuarios con poca formación en informática y programación, sería positivo agrupar todas las funcionalidades desarrolladas en una interfaz gráfica que facilitaría mucho su utilización.

\newpage \thispagestyle{empty} % Página vacía 

\chapter*{Glosario de acrónimos}
\addcontentsline{toc}{chapter}{Glosario de acrónimos}

\begin{itemize}
\item{\textbf{DNA}:  Deoxyribonucleic acid}
\item{\textbf{RNA}:  Ribonucleic acid}
\item{\textbf{CDS}:  Coding Sequence}
\item{\textbf{GFF}: General Feature Format}
\item{\textbf{FCS}: Food Contact Surface}
\item{\textbf{NFCS}: No Food Contact Surface}
\item{\textbf{PCR}: Polymerase Chain Reaction}

\end{itemize}

\newpage \thispagestyle{empty} % Página vacía

\addcontentsline{toc}{chapter}{Bibliografía}    %Agregamos al índice el capitulo de bibliografía 

\bibliographystyle{unsrt}   %plain pero ordenado en orden de aparacicion en documento principal
\bibliography{bibliografia}

\appendix   %Indicamos que lo que viene a continuaci�n son ap�ndices

%\frontmatter %Para poner los anexos en numeros romanos

\chapter{Manual de utilización}
\label{Anexo:manualuso}
En este apartado se detallarán todos los aspectos necesarios para la ejecución y el correcto funcionamiento del proyecto. Se guiará al usuario a través de todo el proceso, desde la instalación del software necesario hasta todas las utilidades que tienen que ver con el workflow, pasando por una breve guía de uso de \textit{Galaxy} orientada a nuestro caso.
\section{Requisitos}

\begin{table}[!h]

\begin{center}
\begin{tabularx}{\textwidth}{bs}
\arrayrulecolor{NavyBlue}\hline
\multicolumn{2}{l}{%
\textbf{\textcolor{NavyBlue}{Requisitos hardware}}}\\
\quad 15 GB de espacio libre en disco \\

\quad 16 GB de memoria RAM \\

\quad procesador de 4 núcleos a 2.5 Ghz (recomendado) \\

\arrayrulecolor{NavyBlue}\hline
\multicolumn{2}{l}{%
\textbf{\textcolor{NavyBlue}{Requisitos software}}} \\
\quad Sistema \textit{Ubuntu} (recomendada versión cercana a la 18.04) \\

\quad \textit{Docker} 18.09.1 o superior\\

\hline

\end{tabularx}
\end{center}
\caption{Requisitos del sistema}
\label{table:Requisitos sistema}
\end{table}
En primer lugar, debido a la complejidad de las ejecuciones y al gran tamaño de la imagen \textit{Galaxy} y de los datos de entrada de las ejecuciones, se requiere de unos componentes hardware relativamente exigentes, como puede observarse en el cuadro \ref{table:Requisitos sistema}. Una de las herramientas utilizadas en el workflow, \textit{SPAdes}, necesita cargar gran cantidad de datos en memoria RAM, por lo que se recomienda una capacidad de 16 GB. Se requiere, también, de 20 GB de espacio libre en disco para almacenar cómodamente la imagen \textit{Galaxy} y los datos que se utilicen. El procesador es un aspecto menos restrictivo, pero las ejecuciones serán más lentas con peores procesadores. Como orientación, en un procesador de 8 núcleos a 3.6 Ghz, una ejecución con unos 10 GB de datos de entrada, tarda aproximadamente 3 horas.

En cuanto al software~\ref{table:Requisitos sistema}, el primer requisito es un sistema operativo Linux. Preferiblemente una versión cercana a \textit{Ubuntu 18.04.1}, la utilizada para el desarrollo.

También será necesario disponer de \textit{Docker} en el equipo. Si no está instalado, en el apartado de <<Instalación de \textit{Docker}>> a continuación, se detallará el proceso a realizar.

\section{Utilización}
\subsection{Instalación de Docker}
Partiendo de un sistema \textit{Ubuntu}, lo primero que deberemos instalar será \textit{Docker}. Para ello, abriremos la terminal de \textit{linux} y seguiremos el tutorial de su web oficial\footnote{\url{https://docs.docker.com/install/linux/docker-ce/ubuntu/}}. El primer paso será actualizar la lista de paquetes disponibles para la instalación. Para ello, escribiremos en la terminal:
    \begin{lstlisting}[language=bash]
    $ sudo apt-get update
    \end{lstlisting}
A continuación, instalaremos varios paquetes que serán necesarios más adelante.
    \begin{lstlisting}[language=bash]
    $ sudo apt-get install \
    apt-transport-https \
    ca-certificates \
    curl \
    gnupg-agent \
    software-properties-common
    \end{lstlisting}
Ahora añadiremos la clave para el repositorio oficial de \textit{Docker} al sistema.
    \begin{lstlisting}[language=bash]
    $ curl -fsSL https://download.docker.com/linux/ubuntu/gpg | \ 
    sudo apt-key add -
    \end{lstlisting}
Necesitaremos añadir el repositorio \textit{Docker}.
    \begin{lstlisting}[language=bash]
    $ sudo add-apt-repository \
   ``deb [arch=amd64] https://download.docker.com/linux/ubuntu \
   $(lsb_release -cs) \
   stable''
    \end{lstlisting}
Antes de finalizar, actualizaremos de nuevo la lista de paquetes disponibles.
    \begin{lstlisting}[language=bash]
    $ sudo apt-get update
    \end{lstlisting}
Finalmente, instalaremos \textit{Docker}.
    \begin{lstlisting}[language=bash]
    $ sudo apt install docker-ce
    \end{lstlisting}
Dado que realizaremos frecuentemente ejecuciones con el comando <<docker>>, añadiremos nuestro usuario al grupo de \textit{Docker} para evitar introducir la clave en cada llamada.
    \begin{lstlisting}[language=bash]
    $ sudo usermod -aG docker ${USER}
    \end{lstlisting}
Para hacerlo efectivo, reiniciaremos nuestra sesión de usuario.
    \begin{lstlisting}[language=bash]
    $ su - ${USER}
    \end{lstlisting}
De esta manera, podremos dar por finalizada la instalación de \textit{Docker}.
\subsection{Descarga de la imagen \textit{Galaxy}}
En la carpeta <<exes>> del proyecto, encontraremos todo lo necesario para interactuar con \textit{Docker} y \textit{Galaxy}. En este caso, para la instalación de la imagen, utilizaremos el script <<run.sh>> que se encargará de buscar la imagen \textit{Galaxy} en nuestro ordenador, descargarla, en caso de que no la encuentre, y desplegarla en un contenedor \textit{Docker}. Para ello, desde la terminal nos situaremos en el directorio <<exes>> y escribiremos:
    \begin{lstlisting}[language=bash]
    $ ./run.sh
    \end{lstlisting}
Tras la ejecución de ese comando, en la terminal debería mostrarse un registro del funcionamiento de \textit{Galaxy}, que podremos cerrar cuando deseemos. Si se quiere comprobar el estado de los contenedores \textit{Docker}, se ejecutará
\begin{lstlisting}[language=bash]
    $ docker ps -a
\end{lstlisting}
Dado que \textit{Galaxy} ya está funcionando, podremos acceder simplemente abriendo cualquier navegador web y escribiendo en la barra de navegación <<\url{http://localhost:8080/}>>.


\subsection{Control del contenedor \textit{Docker}}
En el mismo directorio <<exes>> en el que nos situábamos en el paso anterior encontramos varias utilidades para el control de \textit{Docker}.
\begin{description}
    \item[run.sh] despliega la imagen de \textit{Galaxy} en un contenedor \textit{Docker}. Sobrescribirá la imagen \textit{Galaxy} actual si existe alguna.
    \item[stop.sh] detiene el contenedor \textit{Docker}.
    \item[start.sh] arranca el contenedor si está detenido. Para iniciar \textit{Galaxy}, a excepción de la primera vez, deberá usarse este script.
    \item[remove.sh] elimina el contenedor.
    \item[purge.sh] elimina ficheros temporales de \textit{Docker} que pueden causar problemas de espacio.
\end{description}


\subsection{Funcionamiento del workflow en \textit{Galaxy}}
Aunque existe una utilidad para simplificar el proceso de uso de workflow, explicada en la sección \ref{EjecucionSimplificada}, a continuación analizaremos el uso de \textit{Galaxy} de la manera estándar.

Desde un navegador web cualquiera, si accedemos a la dirección <<\url{http://localhost:8080/}>>, veremos la página de inicio de \textit{Galaxy}. En ella, tendremos la posibilidad de identificarnos mediante la pestaña <<Login or Register>>. Existe un usuario preestablecido con correo <<admin@galaxy.org>> y contraseña <<admin>>. Una vez identificados como administrador tendremos la posibilidad de acceder a los workflows almacenados desde la pestaña <<Workflow>>. En ella, veremos un listado con todas las opciones posibles, la que nos interesa es <<CJ\_Workflow>>. Si desplegamos utilizando la flecha a la derecha del nombre, podremos acceder a las acciones con las que interaccionaremos con este workflow. La primera de ellas es el modo edición, a través de <<edit>>. 

\begin{figure}[!h]
    \begin{center}
      \includegraphics[scale=0.55]{images/SubidaDatasets.png}
      \caption{Interfaz de subida de los conjuntos de datos}
      \label{fig:SubidaDatasets}
    \end{center}
\end{figure}

En la vista de edición, representada en la figura \ref{fig:SubidaDatasets}, encontraremos un diagrama formado por las herramientas y uniones que indican las interacciones entre los ficheros de cada una de ellas. Al comienzo del proceso veremos dos cajas con el título <<Input dataset collection>>. Desde este paso se introducirán los datos de entrada al workflow en forma de una estructura denominada colección. Para ver cómo crear esa estructura saldremos un momento de la pestaña workflow, haciendo clic en la opción <<Analyze data>> de la parte superior. A continuación, en el listado de categorías de herramientas que podemos ver en la parte izquierda, abriremos <<Get Data>> seguido de <<Upload File>>. Veremos como se despliega una ventana nueva en la que existen tres opciones de estructuras: <<regular>>, <<composite>> y <<collection>>. Deberemos seleccionar esta tercera opción y <<Choose local files>> para subir todos los ficheros del primer sentido de las secuencias. Es importante que los subamos en el mismo orden en el que lo vayamos a hacer con la colección de las secuencias en el sentido opuesto. Una vez seleccionados deberemos pulsar <<Start>> y, cuando las barras de progreso estén completas, <<build>>. Esto nos llevará a otra ventana en la que seleccionaremos el nombre de la colección con las lecturas en este sentido. Para finalizar, pulsaremos <<create list>>. Con este proceso tendremos una de las colecciones, y deberemos repetirlo para la colección con las lecturas en el sentido contrario. Cuando se hayan completado todos estos pasos, podremos volver a la vista de edición del workflow para seguir entendiendo su funcionamiento.


\begin{figure}
    \begin{center}
      \includegraphics[scale=0.3]{images/InterfazWorkflowEdit.png}
      \caption{Interfaz de edición del workflow}
      \label{fig:WorkflowEdit}
    \end{center}
\end{figure}
Una vez visto el proceso de introducción de datos de entrada, encontramos las herramientas del workflow, cuyo funcionamiento se explica con más profundidad en el \autoref{chap:sistemadesarrollado}: <<Sistema, Diseño y Desarrollo>>. Para modificar los parámetros de cualquiera de las herramientas, podremos seleccionarla y veremos una serie de opciones en la parte derecha de la pantalla, como vemos en la imagen \ref{fig:WorkflowEdit}. Todos los parámetros modificados ya están guardados en el workflow, pero si se desea cambiar alguno de ellos, se podrá hacer desde esa interfaz. En caso de que se necesite añadir alguna herramienta extra, en el listado de la izquierda se encuentran ordenadas por categorías y con un clic serán añadidas, a falta de enlazar sus ficheros de entrada y de salida con las demás herramientas. Otro aspecto relevante a destacar es que, por defecto, en \textit{Galaxy} los datos se establecen como ocultos a no ser que se marquen con el asterisco que podemos ver a la derecha de los nombres de cada conjunto de datos de salida. Si hacemos clic en ese símbolo, los ficheros ya no aparecerán como ocultos y serán visibles en nuestro historial.

Los historiales son el sistema con el que \textit{Galaxy} ordena los conjuntos de datos del usuario. Desde la página de inicio <<Analyze Data>> podremos consultar el historial activo en este momento  en la parte derecha de la pantalla, pero este no tiene por qué ser el único que existe. Para consultar todos los historiales, desde la propia columna del historial activo, en la parte superior derecha, veremos tres símbolos con las opciones <<Refresh history>>, <<History options>> y <<View all histories>>. Desde esta última accederemos a una nueva pantalla en la que podremos consultar todos los historiales. Es posible que queramos ver conjuntos de datos ocultos de un historial, para ello, podremos activar la opción <<show hidden datasets>> situada debajo del nombre del historial.

Una vez comprendido el funcionamiento del workflow y los historiales, podremos ejecutar el workflow. Para ello, accederemos de nuevo a la pestaña <<Workflow> y desde el desplegable de <<CJ\_Workflow>> seleccionaremos la opción <<run>>. Esto nos situará en una nueva pantalla en la que, de nuevo, podremos modificar todos los parámetros de las herramientas del workflow. Sin embargo, es algo opcional en este momento. El punto principal es la introducción de los datos de entrada, que se seleccionarán desde dos desplegables (uno para cada sentido de las lecturas) en los apartados <<Input dataset collection>>, indicados en la imagen \ref{fig:RunInputs}. Si hemos creado las colecciones como se ha explicado previamente, estas aparecerán en el desplegable y solo tendremos que seleccionarlas. El paso siguiente, teniendo en cuenta que no queremos modificar ningún parámetro, será la ejecución del workflow usando el botón <<Run Workflow>> de la parte superior derecha.

A través del uso de los historiales, tendremos la opción de consultar el estado de las ejecuciones de cada una de las tareas del workflow. Teniendo en cuenta los pasos para consultar los historiales explicados anteriormente, podremos ver si una tarea se está ejecutando, en cola, pausada o si ha  causado algún error. Hay que tener en cuenta que es una ejecución computacionalmente muy costosa, por lo que puede llevar varias horas completarla. Por ello, es importante seguir el progreso a través del historial con el objetivo de asegurarnos de que no ha habido ningún error durante el proceso. Cuando todas las tareas se muestren en color verde, el workflow habrá completado su ejecución y podremos descargar los conjuntos de datos resultantes que nos interesen haciendo clic en ellos y seleccionando el icono con la opción <<download>>.

\begin{figure}
    \begin{center}
      \includegraphics[scale=0.5]{images/RunInputs.png}
      \caption{Selección de las colecciones de entrada del workflow}
      \label{fig:RunInputs}
    \end{center}
\end{figure}

\subsection{Utilidad de ejecución simplificada}
\label{EjecucionSimplificada}
Para facilitar el proceso explicado anteriormente, se ha desarrollado una utilidad que automatiza todos los pasos descritos. En el directorio <<exes>> del proyecto, encontramos un ejecutable llamado <<GalaxyWorkflow>>. Este programa es capaz de cargar los datos de entrada, crear las colecciones, ejecutar el workflow y descargar los resultados en una carpeta local sin necesidad de que el usuario interaccione con \textit{Galaxy}. 

Es necesario conocer algunos comandos para navegar y realizar operaciones básicas con la terminal. En este caso solo veremos los comandos <<ls>>, <<cd>> y <<cp>>, pero es muy recomendable consultar la guía de \textit{Ubuntu} que incluye otros comandos básicos que pueden ser útiles~\footnote{\url{https://help.ubuntu.com/community/UsingTheTerminal}}. Supongamos que nos encontramos en el directorio de nuestro usuario, la ruta por defecto en la que se abrirá nuestro terminal. Si escribimos <<ls>> obtendremos un listado de los directorios y ficheros que nos podemos encontrar, como podemos observar en la imagen \ref{fig:ls}. Tanto para este comando como para cualquier otro, podremos obtener más información sobre su uso añadiendo <<man>> antes del comando, es decir: <<man ls>>. Si ahora nuestro objetivo es movernos a uno de esos directorios que aparecen en el listado, ejecutaremos <<cd>> y el directorio al que queremos movernos. Por ejemplo, si fuese el escritorio: <<cd Desktop>>. En caso de que necesitemos volver al directorio padre del que nos encontramos ahora mismo usaremos <<cd ..>>.

En principio, se pueden copiar los ficheros directamente utilizando la interfaz gráfica de \textit{Ubuntu} al directorio que deseemos, pero si se quiere usar la terminal, el comando necesario para copiar será <<cp>> y para mover <<mv>>, que funcionan de la misma manera. Un ejemplo para copiar un fichero <<cepa200.fastq>> al escritorio sería
\begin{lstlisting}[language=bash]
    $ cp cepa200.fastq ./Desktop/cepa200.fastq
\end{lstlisting}
y para moverlo 
\begin{lstlisting}[language=bash]
    $ mv cepa200.fastq ./Desktop/.
\end{lstlisting}
Si se quisiese hacer lo mismo para todos los ficheros de determinada extensión, por ejemplo, fastq: 
\begin{lstlisting}[language=bash]
    $ cp *.fastq ./Desktop/
\end{lstlisting}

Por último, veamos como arrancar un ejecutable. Si suponemos que en el escritorio existe un fichero <<run.sh>>, deberemos escribir la ruta hasta ese fichero precedida de un punto. Por ejemplo, si estuviésemos en el directorio comentado anteriormente, ejecutaríamos <<run.sh>> escribiendo
\begin{lstlisting}[language=bash]
    $ ./Desktop/run.sh
\end{lstlisting}
 Si nos encontramos en el mismo directorio que el ejecutable, la ruta hasta ese fichero será simplemente un punto: 
 \begin{lstlisting}[language=bash]
    $ ./run.sh
\end{lstlisting}

\begin{figure}[!h]
    \begin{center}
      \includegraphics[scale=0.8]{images/ls.png}
      \caption{Ejemplo de un comando ls}
      \label{fig:ls}
    \end{center}
\end{figure}

Pasemos ahora a la utilización del propio workflow. En primer lugar, se requiere que la imagen \textit{Galaxy} esté funcionando y podamos acceder a ella desde el navegador en la dirección <<\url{http://localhost:8080/}>>. Es conveniente realizar esta comprobación antes de ejecutar la aplicación. Si se desea, el usuario puede identificarse como administrador antes de comenzar, para poder consultar lo que está ocurriendo con \textit{Galaxy} mientras la utilidad está funcionando. Es importante dejar claro que interaccionará con dos nuevos historiales, así que para comprobar el funcionamiento, si es que se desea, se deberá hacer desde la vista de todos los historiales.

El proceso para ejecutarlo es muy sencillo, solamente deberemos abrir la terminal de \textit{Ubuntu} y situarnos en el directorio <<exes>> del proyecto. Una vez ahí, veremos que existen dos carpetas <<Forward>> y <<Reverse>>. En ellas deberemos introducir los ficheros \textit{.fastq} de las lecturas que deseamos analizar. Los ficheros de cada sentido de lectura deben dividirse en las dos carpetas. Es necesario que ambas tengan el mismo número de ficheros y muy recomendable que los nombres de los dos sentidos de una lectura comiencen por el mismo identificador único. Por ejemplo, dos nombres de una pareja podrían ser  <<lectura\textbf{1}\_forward>> y <<lectura\textbf{1}\_reverse>>. Cuando contemos con todos los ficheros situados en los directorios correspondientes, escribiremos el siguiente comando.
\begin{lstlisting}[language=bash]
    $ ./GalaxyWorkflow
\end{lstlisting}
Con ello, comenzará la ejecución y cada minuto recibiremos una actualización del estado de las tareas en marcha, pudiendo comprobar si ha habido algún error y ser capaz de detener la ejecución desde \textit{Docker}. Si esto sucediera, la manera más fácil sería detener la imagen \textit{Galaxy} con el comando:
\begin{lstlisting}[language=bash]
    $ ./stop.sh
\end{lstlisting}
O, incluso, resetear la imagen de cero para eliminar todos los ficheros que ya no nos interesan, con la combinación de comandos
\begin{lstlisting}[language=bash]
    $ ./stop.sh
    $ ./remove.sh
    $ ./run.sh
\end{lstlisting}
Normalmente, esto no será necesario y, tras varias horas de ejecución, obtendremos los resultados en un directorio <<ResultsHistory[+timestamp]>> en el que serán ordenados en carpetas dependiendo de la herramienta de la que provengan.

\subsection{Tratamiento de datos obtenidos}
Con la utilidad del apartado anterior, obtenemos un gran número de ficheros independientes. El objetivo de esta herramienta es agrupar los resultados de \textit{ABRicate} y \textit{Roary} en un solo documento \textit{.xls} que podamos abrir como una hoja de cálculo. Para ello, deberemos situarnos de nuevo en el directorio <<exes>>. Nuestros datos de entrada estarán situados en una carpeta con un nombre que comience por <<Results>> y que contenga varias subcarpetas, cada una con el nombre de la herramienta de origen de los ficheros que contiene. Por ejemplo, dos carpetas: <<roary>> y <<abricate>> con los ficheros resultantes de cada una en su interior. Una vez tengamos esta estructura (que por defecto se cumple si se utiliza la herramienta de ejecución automática del workflow) podremos abrir la terminal en el directorio <<exes>> e introducir el comando
\begin{lstlisting}[language=bash]
    $ ./OutputToXls
\end{lstlisting}
Esto generará un fichero con el mismo nombre que la carpeta que hemos indicado anteriormente pero con extensión \textit{.xls} en el que cada hoja del documento corresponderá a una de las herramientas.
\newpage \thispagestyle{empty} % Página vacía 

\chapter{Manual del programador}
\label{Anexo:ManualProgramador}
Este capítulo tiene como objetivo facilitar el acercamiento al proyecto por parte de un programador externo. Siguiendo los apartados desarrollados a continuación, tendrá la información necesaria para realizar modificaciones o ampliar el trabajo ya realizado.

\section{Requisitos hardware recomendados}
Debido a que se realizan ejecuciones muy costosas en local, los requisitos recomendados para este proyecto son bastante exigentes.
En primer lugar, dado que las imágenes \textit{Galaxy} son muy pesadas (del orden de los 10 GB) y los datos generados por el workflow también son considerables, se recomienda un espacio libre en disco de unos 20 GB para una utilización cómoda de la aplicación.
En cuanto a la memoria RAM, se recomienda utilizar una capacidad mayor que el tamaño de los datos de entrada. Esto se debe a la forma de ejecución de algunas herramientas del workflow, que necesitan cargar todas las secuencias a la vez en memoria. En nuestro caso se han utilizado 16 GB de RAM para un conjunto de datos de entrada de 10 GB.
El caso del procesador no es tan limitante como el anterior, sin embargo, definirá el tiempo total de ejecución. El caso de prueba mencionado anteriormente se ha ejecutado con un procesador Intel i7-9700K, con un tiempo de ejecución de unas 3 horas. Por lo tanto, cualquier procesador de un sistema de 64 bits será capaz de realizar la ejecución.

\section{Requisitos software necesarios}
El proyecto está enfocado a ser ejecutado en un sistema \textit{Ubuntu}, en concreto, se ha utilizado la versión 18.04.1. Sin embargo, desde la perspectiva de un desarrollador, con unos cambios menores puede llegar a adaptarse a cualquier sistema sin demasiado esfuerzo.
    \subsection{\textit{Docker}}
    \textit{Docker} es un sistema de código abierto basado en contenedores que permiten el despliegue de aplicaciones dentro de sí mismos. Los contenedores ofrecen la posibilidad de empaquetar todo el software necesario para ejecutar nuestra aplicación, ofreciendo un modo de virtualización mucho más aislado y ligero que el uso de máquinas virtuales. Además, se mantiene la seguridad de que nuestra aplicación será utilizable en cualquier sistema en el que se pueda realizar esta virtualización.
    
    Durante el desarrollo del proyecto se ha utilizado la versión 18.09.1 de \textit{Docker}, por lo que se recomienda la instalación de una versión igual o superior a esta. El proceso para instalar \textit{Docker} en \textit{Ubuntu} no es complejo y en la propia web encontramos una guía de instalación \footnote{\url{https://docs.docker.com/install/linux/docker-ce/ubuntu/}}.


    \subsection{Python}
    \textit{Python} es uno de los lenguajes de programación más utilizados en el mundo. Es un lenguaje interpretado, de alto nivel, de propósito general, de código abierto y multiplataforma. Su facilidad de uso y comprensión han hecho que su crecimiento haya sido notable en los últimos años, especialmente en investigación.
    Algunas de las funcionalidades del desarrollo se han creado a través de \textit{Python 3.7.2}. Se recomienda la utilización de una versión superior a 3.6. Además de \textit{Python} estándar, se han utilizado varias librerías:
        \begin{description}
            
            \item[\textit{Bioblend} 0.12.0] es un paquete que posibilita la interacción con la API de \textit{Galaxy}, permitiendo actuar sobre la instancia desplegada en \textit{Docker}.
            
            \item[\textit{Pandas} 0.24] permite la utilización de nuevas estructuras y métodos para el análisis de datos. Se utiliza en este caso para tareas simples, como gestionar ficheros de tipo \textit{tabular}, \textit{csv} y \textit{xls}.
            
            \item[\textit{Numpy} 1.15.4] es uno de los paquetes más utilizados de \textit{Python}, facilitando gran variedad de funciones matemáticas. En el caso de \textit{Numpy} es importante utilizar esta versión en concreto, ya que en este momento la versión actual (1.16.0) presenta incompatibilidades con otra de las librerías instaladas, \textit{PyInstaller}.
            
            \item[\textit{PyInstaller} 3.4] es una utilidad que permite generar ejecutables a partir del código \textit{Python} para así evitar las instalaciones, tanto del propio \textit{Python}, como de sus dependencias.
        
        \end{description}
    
        Las instalaciones de todas estas librerías pueden realizarse sin problema desde \textit{pip} con el comando
        \begin{lstlisting}[language=bash]
        $ pip install <<libreria_deseada>>.
        \end{lstlisting} 
        En caso de que se necesite instalar una versión concreta, como en \textit{Numpy}, se utilizará
        \begin{lstlisting}[language=bash]
        $ pip install numpy==1.15.4
        \end{lstlisting} 

\section{Estructura del proyecto}
\begin{figure}
    \begin{center}
      \includegraphics[scale=0.4]{images/FolderStructure.png}
      \caption{Contenido del proyecto}
      \label{fig:ContenidoDelProyecto}
    \end{center}
\end{figure}

\subsection{Directorio de ensamblado: \textit{<<build>>}}
Todos los componentes necesarios para construir la imagen \textit{Docker} de \textit{Galaxy} se encuentran en el directorio <<build>>. El fichero más relevante es <<Dockerfile>>, que contiene todas las directivas que indican cómo construir esta imagen. En concreto, realiza algunas variaciones sobre una imagen ya existente. Para realizar este montaje, será necesario el fichero <<mount.sh>>, que se encargará de utilizar las instrucciones de <<Dockerfile>> para construir y desplegar la imagen en un contenedor \textit{Docker}. Los dos ficheros <<install\_workflows\_wrapper.sh>> y <<my\_tool\_list.yml>> son utilidades para la instalación de workflows y herramientas de \textit{Galaxy} respectivamente. Las herramientas indicadas se instalarán desde el <<tool shed>> de \textit{Galaxy}, mientras que los workflows se encontrarán en el directorio <<workflows>>.

\subsection{Directorio de desarrollo: \textit{<<dev>>}}
Las utilidades desarrolladas en \textit{Python} se almacenan en este directorio. Aquí encontramos dos ficheros principales, <<GalaxyWorkflow.py>> y <<OutputToXls.py>. El primero de ellos es el encargado de simplificar la utilización del workflow a los usuarios. Basándose en la API de \textit{Galaxy} a través del uso de la librería \textit{BioBlend}, se identifica en \textit{Galaxy} utilizando las credenciales indicadas en el fichero <<credentials.py>>. A continuación, genera nuevos historiales para almacenar los datos de entrada y los futuros datos de salida. Buscará dentro de los directorios <<Forward>> y <<Reverse>> los ficheros de las secuencias a analizar. Con ellos creará dos colecciones que servirán como entrada definitiva del workflow. Una vez ejecutado, el script se encargará de descargar los ficheros de salida de las tres últimas herramientas, que son los que en este caso se desean utilizar posteriormente.

Por su parte, el script <<OutputToXls.py>> se encargará, a través del uso de \textit{Pandas}, de crear un solo fichero <<.xls>> a partir de las salidas generadas por las herramientas \textit{Roary} y \textit{ABRicate} para facilitar el tratamiento de los datos por parte del equipo que los va a utilizar, dado que están familiarizados con \textit{Excel}.

\subsection{Directorio de ejecución: \textit{<<exes>>}}
En este directorio encontraremos, en primer lugar, los ejecutables correspondientes a los ficheros \textit{Python} del directorio <<dev>>, con el objetivo de evitar la instalación de \textit{Python} y sus dependencias al usuario. Además, se han desarrollado varios ficheros para simplificar el manejo de la imagen \textit{Galaxy} de \textit{Docker}, que simplemente contienen comandos para su instalación, arranque, parada y borrado. Se incluye un script <<purge.sh>> debido a que son frecuentes las acumulaciones de ficheros basura que acaban por ocupar demasiado espacio de almacenamiento. Este script es ejecutado desde los demás para evitar estos problemas.

\newpage \thispagestyle{empty} % Página vacía 

%Hoja final en blanco
\newpage \thispagestyle{empty} % P�gina vac�a

\end{document}
